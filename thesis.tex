\documentclass[12pt,letterpaper]{article}
\usepackage[utf8]{inputenc}
\usepackage[T1]{fontenc}
\usepackage{amsmath}
\usepackage{amsfonts}
\usepackage{amssymb}
\usepackage{graphicx}
\usepackage{setspace}
\usepackage{palatino}
\usepackage[notes,backend=biber]{biblatex-chicago}
\bibliography{thesis.bib}
\usepackage[left=1.20in, right=1.20in, top=1.20in, bottom=1.20in]{geometry}
\begin{document}
	
	\title{The Classical Musical Educiational Project of \textit{Sesame Street}}
	\author{Eden Attar}
	\maketitle
	
	\doublespacing
	\thispagestyle{empty}
	
	\section*{Introduction: In \textit{media} res}
	
	\noindent
	On a summer evening in Saint Louis I was listening to the radio in
	the kitchen. In the research for this paper, which had grown from a character
	study on Yo-Yo Ma to encompass the themes of eduction and public broadcasting,
	I had decided that I should actually own some kind of broadcast receiver. 
	After looking into mini televisions, I settled on the conservative option
	of a radio. As I chopped onions, my local NPR station's weekend programming
	buzzed in the background.
	
	A program on the recent events in Afghanistan had ended and the PSA 
	space, where a commercial station
	would have run adds, began. I heard a woman's voice say ``STLPR: understanding
	starts here.'' 

	My ears perked up. What an intriguing declaration. It is not ``STLPR:
	your source of news,'' or ``STLPR: hear the nation.'' Instead, the 
	slogan is a 
	promise to increase your \textit{understanding. }NPR is offering not
	just to 
	inform you, but moreover to provide some kind of ethical education,
	presumably into cultural-socio-political events such that one comes away
	\textit{understanding.}

	A strange word, ``understanding.'' In the context of the socio-political,
	cultural and economic topics under the purview of national public 
	radio, something seems very kind about the word. It feels emotionally
	laden and steeped in empathy. 

	Its usage does not imply factual understanding, something like ``Let
	us factually explain what's going on in Afghanistan,'' but rather, 
	``cultivate an understanding of the human experiences of Afghanistan 
	along with us.'' 


	Etymologically, the ``under'' in the first half of the word 
	``understand'' does not denote the more modern ``below,'' but rather 
	among or in-between. \autocite{Ety} To understand is to stand among. 
	What \textit{NPR} is offering, then, is empathy in a broadcast. 


	This is continuous with the founding rhetoric of \textit{NPR} and its
	focus on hearing common people speak. 
	\textit{All Things Considered}, now one of the most listened-to programs
	on American radio, debuted with the chaotic coverage of
	what at the time was the biggest anti-Vietnam war protest to date. It
	consisted of tape gathered by multiple reporters at the event with no 
	overarching narrative. It was unlike anything on the radio at the time.
	Its chaos was intended to capture the chaos of the moment, but was also
	part of a larger aesthetic and political trend on the program to hear 
	the voices of normal Americans, which is now known as the 
	``public radio sound.'' 

	The sound is recognizable to this day by its lack of a 
	narrator and the featuring of the voices of common people. It 
	was pioneered by the ``founding mother'' of \textit{NPR}, radio 
	journalist Susan Stamberg. She was inspired by the cinema verité of the
	time, saying that the idea ``was to tell a story without telling it---
	just sound through chunks.'' 
	One particular aspect of the public radio sound was the use of ``cross
	talk,'' a technique in which two voices are layered onto one another 	
	in polyphony, which was inspired by Glenn Gould's ``contrapuntal 
	storytelling'' in \textit{The Solitude Trilogy.}\autocite[197]{Porter}
	Stamberg retrospectively described the debut as having ``no interfering 
	narrator.''\autocite[185]{Porter}

	The showcasing of alternative voices was so central to the ethos of the
	young network that \textit{NPR} turned down a \$300,000 donation from 
	the Ford 
	Foundation which was attached to the notwork hiring well-known broadcast
	journalist Edward P. Morgan as host of \textit{All Things Considered}. 
	Morgan, holder of a Peabody Award, would have certainly 
	brought the new network
	credibility, but \textit{NPR} feared that the authoritative voice would
	get in the way of the alternative voices that they wanted to showcase.
	
	The ethos of ``standing among'' seems to be baked in to the entire
	project of public radio.


	
\newpage
%\singlespacing
\printbibliography
\end{document}
