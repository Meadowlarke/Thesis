\documentclass[12pt,letterpaper]{article}
\usepackage[utf8]{inputenc}
\usepackage[T1]{fontenc}
\usepackage{amsmath}
\usepackage{amsfonts}
\usepackage{amssymb}
\usepackage{graphicx}
\usepackage{setspace}
\usepackage{palatino}
\usepackage[notes,backend=biber]{biblatex-chicago}
\bibliography{thesis.bib}
\usepackage[left=1.20in, right=1.20in, top=1.20in, bottom=1.20in]{geometry}
\begin{document}
	
	\title{Playing \textit{with} the Music: Ecologies of Attention and Understating 
	in the Classical Music Education Project of \textit{Sesame Street}}
	\author{Eden Attar}
	\maketitle
	\doublespacing
	\thispagestyle{empty}
	\newpage

	\section*{Introduction}	

	If I have done anything right this paper should be easy to read. I
	think that is the kindest thing a writer can do. If I have
	done anything right this paper should also not be quite like anything 
	you have seen before. I might say that is the second kindest thing a 
	writer can do.  

	In the spirit of easy reading, here's a summary of what lies ahead:

	This is a paper about the ecologies of attention and understanding 
	evident in and emergent from the presentation of classical music on the
	classic children's television program \textit{Sesame Street.}  
	
	The paper takes three parts. The first part is an analysis of context,
	attention, and education on the program, framed through the analogy of
	bird-watching. I argue that the classical music education which is 
	presented on the program deviates from the humanitarian norm of 
	``uplift'' and absolute aesthetics---the bird's-eye view---and instead
	presents an understanding of music which resonates with ideas of 
	``listening with''---the view of the bird watcher, or in the words of
	Jenny Odell, the bird \textit{listener} who observes and is observed;
	who through acts of attention learns to perceive the world around them
	pluralisticly. In this section I argue that while \textit{Sesame
	Street} engages with aspects of critical pedagogy, it is ultimately 
	resistant to educational classifications. The program is itself
	\textit{ecological}, composed of many interacting and competing parts
	which create a program whose character is emergent and consistent, but
	not unified.   

	Part two is an analysis of classical music performance on the program, 
	framed through the analogy of the worm-listener---they who listen as 
	outsiders to 
	the arcane sound of the world beneath our feet. This section deals with
	questions of parody, virtuosity, and childhood. It contains three
	character studies of classical musicians who made appearances on the
	program: Yo-Yo Ma, Evelyn Glennie, and Lang Lang.   

	In this section I argue that \textit{Sesame Street} takes part in a
	long tradition of classical music parody which welcomes the outsider
	while at the same time engages the classical music native and initiate
	---that is, the worm, who has spent its life embedded in the classical
	strata.
	\textit{Sesame Street} neither challenges nor exalts the classical 
	music canon, but rather serves as an enabling technology, a 
	\textit{geophone}, which allows for the areal listener to perceive and  
	engage with the arcane sounds of underworld.  

	This project takes shape in two media: one is what you are reading 
	right now, the 
	``white paper'' presentation of the research. The other is a 
	non-striated recreation of the field in the form of a Multi User 
	Domain, or MUD. The third section of this paper deals with the themes
	of diversity,
	equity, and liberation which inform the project.  

	Each part begins with a prologue---or dare I say, a dramatic monologue
	---written in first-person perspective
	which grounds the following discussion. If the chapters look at
	musical and educational ecologies, the prologues give lived context for
	the geological bedrock on which these ecologies are based.  

	I'll count myself off off: LIGHTS. CAMERA. ACTION!	

	\newpage	
	\section*{Prologue: In \textit{media} res}
	
	\noindent
	On a summer evening in Saint Louis I was listening to the radio in
	the kitchen. In the research for this paper, which had grown from a character
	study on Yo-Yo Ma to encompass the themes of eduction and public broadcasting,
	I had decided that I should actually own some kind of broadcast receiver. 
	After looking into mini televisions, I settled on the conservative option
	of a radio. As I chopped onions, my local NPR station's weekend programming
	buzzed in the background.
	
	A program on the recent events in Afghanistan had ended and the PSA 
	space, where a commercial station
	would have run adds, began. I heard a woman's voice say ``This is Saint
	Louis Public Radio. Understanding Starts Here.'' 

	My ears perked up. What an intriguing declaration. It is not ``This is 
	Saint Louis Public Radio, your source of news,'' or ``This is Saint 
	Louis Public Radio: hear the nation.'' Instead, the 
	slogan is a 
	promise to increase your \textit{understanding. }NPR is offering not
	just to 
	inform you, but moreover to provide some kind of ethical education,
	presumably into cultural-socio-political events such that one comes away
	\textit{understanding.}

	A strange word, ``understanding.'' In the context of the socio-political,
	cultural and economic topics under the purview of national public 
	radio, something seems very kind about the word. It feels emotionally
	laden and steeped in empathy. 

	Its usage does not imply factual understanding, something like ``Let
	us factually explain what's going on in Afghanistan,'' but rather, 
	``cultivate an understanding of the human experiences of Afghanistan 
	along with us.'' 


	Etymologically, the ``under'' in the first half of the word 
	``understand'' does not denote the more modern ``below,'' but rather 
	among or in-between. \autocite{Ety} To understand is to stand among. 
	What \textit{NPR} is offering, then, is empathy in a broadcast. 


	This is continuous with the founding rhetoric of \textit{NPR} and its
	focus on hearing common people speak. 
	\textit{All Things Considered}, now one of the most listened-to programs
	on American radio, debuted with the chaotic coverage of
	what at the time was the biggest anti-Vietnam war protest to date. It
	consisted of tape gathered by multiple reporters at the event with no 
	overarching narrative. It was unlike anything on the radio at the time.
	Its chaos was intended to capture the chaos of the moment, but was also
	part of a larger aesthetic and political trend on the program to hear 
	the voices of normal Americans. Something now known as the 
	``public radio sound.'' 

	The sound is recognizable to this day by its lack of a 
	narrator and the featuring of the voices of common people. It 
	was pioneered by the ``founding mother'' of \textit{NPR}, radio 
	journalist Susan Stamberg. She was inspired by the cinema verité of the
	time, saying that the idea ``was to tell a story without telling it---
	just sound through chunks.'' This meant no authoritative radio voice,
	but instead just giving everyday Americans time to talk before moving 
	along to another speaker. The work of narrative interpretation is left
	to the listener, who is given a more rhizomatic view of the events---
	a radical new direction in radio journalism and broadcast culture more
	generally.  

	One particular aspect of the public radio sound was the use of ``cross
	talk,'' a technique in which two voices are layered onto one another 	
	in polyphony, which was inspired by Glenn Gould's ``contrapuntal 
	storytelling'' in \textit{The Solitude Trilogy.}\autocite[197]{Porter}
	Stamberg retrospectively described the debut as having ``no interfering 
	narrator.''\autocite[185]{Porter}

	The showcasing of alternative voices was so central to the ethos of the
	young network that \textit{NPR} turned down a \$300,000 donation from 
	the Ford 
	Foundation contingent on the network hiring well-known broadcast
	journalist Edward P. Morgan as host of \textit{All Things Considered}. 
	Morgan, recipient of a Peabody Award, would have certainly 
	brought the new network
	credibility, but \textit{NPR} feared that the authoritative voice would
	get in the way of the alternative voices that they wanted to showcase.
	The ethos of ``standing among'' seems to be baked in to the entire
	project of public radio.

	This ethos is visible in public television as well. In the mid-1960s
	Ralph Lowell, Board Chairman of the Boston Educational Television 
	station proposed that the Oval Office put together a commission on 
	educational television. President Johnson, believing that this was
	a job for the private sector, turned to the Carnegie Corporation.
	What was formed was a diverse team known as the Carnegie Commission
	on Educational Television. Among others, this team included writer Ralph
	Ellison, concert pianist Rudolph Serkin, and labor leader Leonard 
	Woodcock.\autocite[409]{Meany}

	The commission put together a report titled \textit{Public Television, A 
	Program for Action} which stressed the artistic and transformative value
	of the medium and the importance of using it to resist the pressure 
	toward uniformity found in commercial television but instead seek out 
	and satisfy the needs of the nation's diversity.\autocite[410]{Meany}

	The need for this shift in broadcasting came about because the
	United States Government had granted access to the airwaves to 
	commercial stations
	first, only adding public stations to the media landscape as part of
	Great Society era reformations. Canada and England, on the other
	hand, created public broadcasting first and only later opened the
	airwaves to commercial stations.\autocite[412]{Meany}

	It was this same era of reform that lead the Carnegie Corporation
	in search of a way to increase the reach of their 
	humanitarian educational programs for inner-city preschoolers.
	\autocite[15]{Davis} Lloyd Morrisett, the vice president of programs at
	the Carnegie Corporation\autocite[7]{Cooney} and a psychologist
	\autocite[15]{Davis} had noticed his own daughter's fascination with
	the television medium when his three-year old got up early in the 
	morning to watch the pre-broadcast RCA test patterns.
	\autocite[11]{Davis}
	
	Morrisett then tasked Joan Cooney, a publicity specialist for the 
	\textit{United States Steel Hour}, a twice-monthly series on CBS, 
	\autocite[27]{Davis} with the project of producing a study
	on the educational possibilities of children's television, backed by a
	\$15,000 grant. The resulting report, \textit{The Potential Uses of 
	Television in Children's Education} analyzed the ways that the
	relatively new technology of television could be used as an educational
	supplement for preschoolers, particularly those from disadvantaged 
	backgrounds.

	It was this report which lead to the creation of the program at hand, 
	\textit{Sesame Street.}


	\newpage
	\thispagestyle{empty}
	\vspace*{30pt}
	\begin{center}
	{\Huge Part I\\ 
	\Large The Bird-Watcher:\\
	Attention, Education, and Entertainment}
	\end{center}
	\newpage

	\section*{Entertainment: Welcome to \textit{Sesame Street}}

	\noindent\textit{Sesame Street} needs little introduction. One of the
	longest
	running shows in the history of television,\autocite{Brit} it has been a 
	cultural staple of generations of children---and their parents---in the 
	United States and worldwide.
	
	While \textit{Sesame Street} is an educational program, it is notable 
	for its entertainment value. This stems from the realization by the 
	original producers of the program that in order to compete with the 
	non-educational television competition only a turn of the dial away,
	televised preschool education would have to be just as 
	entertaining.\autocite[38]{Cooney} This is especially true given that
	at the program's premiere in 1969, there was nothing like it in the 
	children's media landscape.\autocite[See chapter 3 for a survey of the 
	contemporary children's television landscape]{Davis} This means that 
	there were no models for the production of the program, nor for its
	reception---they could not trust that children knew how to watch
	educational programs.  

	This realization lead to a fundamental characteristic of the production
	structure of the program: rather than hire child development and 
	education specialists to produce the show, they used education research
	to inform decisions made by seasoned TV, radio, and Broadway producers,
	writers, composers, and actors. This decision was informed by the 
	observation
	that children respond well to adult television and were known to recite
	adds for beer and cigarettes, as well as having their attention held by
	adult variety shows such as \textit{Rowan and Martin's Laugh-In.}
	\autocite[16]{Ostrofsky2012}

	But \textit{Sesame Street} is not just entertaining for children. 
	In order to encourage parents to watch the show along with their
	children, a move which would also help keep preschoolers focused,
	the producers aimed to make the show entertaining enough for parents
	to want to watch it along with their children.
	\autocite[294]{Ostrofsky2017}
	\footnote{Mixed-age television targeting likely began on the 
	television program \textit{Captain Kangaroo}. \autocite[46]{Davis}} 
	To this effect you see... and the featuring of musical performers that
	parents would know.
		
	Entertainment, the producers understood, is a tool for
	cultivating attention.
	

	\section*{Attention}	


	\newpage
	\thispagestyle{empty}
	\vspace*{30pt}
	\begin{center}
	{\Huge Part I\\ 
	\Large The Worm-Listener:\\
	Parody, Virtuosity, and Childhood}
	\end{center}


	\newpage
	\section*{Prologue: The Worm Listener}

	I spent something like a quarter of my life on the south side of 
	Chicago. There was a summer during that period where I got into the 
	habit of biking down to the 63rd street beach to comb the sandy shore 
	for pebbles. There were two kinds that interested me: shiny black 
	pieces of anthracite coal, and pale grey honeycombs of steel slag. Some
	thirty miles down the coast, at what now is the Indiana Dunes National 
	Park, the beach is littered with pebbles of all kinds, glittering 
	happily beneath the surf or ice of the ancient lake. Here, however, in 
	the scorched earth left behind by the bonfire that built and destroyed 
	America, the pebbles were largely fossils of that recent past.

	That summer I went on a sound walk organized by the Midwest Society of
	Acoustic Ecology at the former site of the US Steel foundry. A section 
	of the massive industrial campus had been re-wilded and returned to its
	pre-steel state of fresh-water marsh. Some birds chirped, we saw an 
	egret, but it had turned out to be one of those hot summer days where 
	the air is still, heavy, and all the more stifling for its enveloping 
	silence. 

	At some point the walk took us to a part of the wooded hammock where 
	the artists had driven microphones---geophones---into the earth. When 
	my turn came around I put on the headphones and listened. 

	I was immediately made aware of my on weight on the ground beneath me. 
	Any small motion and I would hear myself in the microphone like an 
	earthquake. Any large motion and the bugs, which I now realized were all
	around me, would stop and I would have to wait for some time before they
	started to move again. 

	I had never heard anything like it. It was a sound impossible to 
	describe in terms of our air-based hearing. And I was part of it. The 
	subterranean bugs heard and responded to me just as I to them. Through 
	the assistive technology of contact microphones and digital amplifiers, 
	I was given access to the worm's-ear view of the ever-present world 
	beneath my feet. The sounds were foreign, but in no way unpleasant. Is 
	this the sound of an ant, a worm? I don't know, but I like it. For me as
	 an air-based listener, this was the sonic experience of a complete outsider.	
	
	\newpage
	\section*{Listening \texit{with} the Music: Ecological Humor} 

	
\newpage
%\singlespacing
\printbibliography
\end{document}


%	While the program's most recognizable faces may be the members of its 
%	puppet cast, who hail from Jim Henson's troupe *The Muppets*, music and
%	musicians play an important role in the program's educational mission,
%	as well as its multi-age appeal. 


