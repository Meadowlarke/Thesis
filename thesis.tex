\documentclass[12pt,letterpaper]{article}
\usepackage[utf8]{inputenc}
\usepackage[T1]{fontenc}
\usepackage{amsmath}
\usepackage{amsfonts}
\usepackage{amssymb}
\usepackage{graphicx}
\usepackage{setspace}
\usepackage{palatino}
\usepackage[notes,backend=biber]{biblatex-chicago}
\bibliography{thesis.bib}
\usepackage[left=1.20in, right=1.20in, top=1.20in, bottom=1.20in]{geometry}
\begin{document}
	
	\title{The Classical Musical Educational Project of \textit{Sesame Street}}
	\author{Eden Attar}
	\maketitle
	
	\doublespacing
	\thispagestyle{empty}
	
	\section*{Introduction: In \textit{media} res}
	
	\noindent
	On a summer evening in Saint Louis I was listening to the radio in
	the kitchen. In the research for this paper, which had grown from a character
	study on Yo-Yo Ma to encompass the themes of eduction and public broadcasting,
	I had decided that I should actually own some kind of broadcast receiver. 
	After looking into mini televisions, I settled on the conservative option
	of a radio. As I chopped onions, my local NPR station's weekend programming
	buzzed in the background.
	
	A program on the recent events in Afghanistan had ended and the PSA 
	space, where a commercial station
	would have run adds, began. I heard a woman's voice say ``This is Saint
	Louis Public Radio. Understanding Starts Here.'' 

	My ears perked up. What an intriguing declaration. It is not ``This is 
	Saint Louis Public Radio, your source of news,'' or ``This is Saint 
	Louis Public Radio: hear the nation.'' Instead, the 
	slogan is a 
	promise to increase your \textit{understanding. }NPR is offering not
	just to 
	inform you, but moreover to provide some kind of ethical education,
	presumably into cultural-socio-political events such that one comes away
	\textit{understanding.}

	A strange word, ``understanding.'' In the context of the socio-political,
	cultural and economic topics under the purview of national public 
	radio, something seems very kind about the word. It feels emotionally
	laden and steeped in empathy. 

	Its usage does not imply factual understanding, something like ``Let
	us factually explain what's going on in Afghanistan,'' but rather, 
	``cultivate an understanding of the human experiences of Afghanistan 
	along with us.'' 


	Etymologically, the ``under'' in the first half of the word 
	``understand'' does not denote the more modern ``below,'' but rather 
	among or in-between. \autocite{Ety} To understand is to stand among. 
	What \textit{NPR} is offering, then, is empathy in a broadcast. 


	This is continuous with the founding rhetoric of \textit{NPR} and its
	focus on hearing common people speak. 
	\textit{All Things Considered}, now one of the most listened-to programs
	on American radio, debuted with the chaotic coverage of
	what at the time was the biggest anti-Vietnam war protest to date. It
	consisted of tape gathered by multiple reporters at the event with no 
	overarching narrative. It was unlike anything on the radio at the time.
	Its chaos was intended to capture the chaos of the moment, but was also
	part of a larger aesthetic and political trend on the program to hear 
	the voices of normal Americans. Something now known as the 
	``public radio sound.'' 

	The sound is recognizable to this day by its lack of a 
	narrator and the featuring of the voices of common people. It 
	was pioneered by the ``founding mother'' of \textit{NPR}, radio 
	journalist Susan Stamberg. She was inspired by the cinema verité of the
	time, saying that the idea ``was to tell a story without telling it---
	just sound through chunks.'' This meant no authoritative radio voice,
	but instead just giving everyday Americans time to talk before moving 
	along to another speaker. The work of narrative interpretation is left
	to the listener, who is given a more rhizomatic view of the events---
	a radical new direction in radio journalism and broadcast culture more
	generally.  

	One particular aspect of the public radio sound was the use of ``cross
	talk,'' a technique in which two voices are layered onto one another 	
	in polyphony, which was inspired by Glenn Gould's ``contrapuntal 
	storytelling'' in \textit{The Solitude Trilogy.}\autocite[197]{Porter}
	Stamberg retrospectively described the debut as having ``no interfering 
	narrator.''\autocite[185]{Porter}

	The showcasing of alternative voices was so central to the ethos of the
	young network that \textit{NPR} turned down a \$300,000 donation from 
	the Ford 
	Foundation contingent on the network hiring well-known broadcast
	journalist Edward P. Morgan as host of \textit{All Things Considered}. 
	Morgan, recipient of a Peabody Award, would have certainly 
	brought the new network
	credibility, but \textit{NPR} feared that the authoritative voice would
	get in the way of the alternative voices that they wanted to showcase.
	The ethos of ``standing among'' seems to be baked in to the entire
	project of public radio.

	This ethos is visible in public television as well. In the mid-1960s
	Ralph Lowell, Board Chairman of the Boston Educational Television 
	station proposed that the Oval Office put together a commission on 
	educational television. President Johnson, believing that this was
	a job for the private sector, turned to the Carnegie Corporation.
	What was formed was a diverse team known as the Carnegie Commission
	on Educational Television. Among others, this team included writer Ralph
	Ellison, concert pianist Rudolph Serkin, and labor leader Leonard 
	Woodcock.\autocite[409]{Meany}

	The commission put together a report titled \textit{Public Television, A 
	Program for Action} which stressed the artistic and transformative value
	of the medium and the importance of using it to resist the pressure 
	toward uniformity found in commercial television but instead seek out 
	and satisfy the needs of the nation's diversity.\autocite[410]{Meany}

	The need for this shift in broadcasting came about because the
	United States Government had granted access to the airwaves to 
	commercial stations
	first, only adding public stations to the media landscape as part of
	Great Society era reformations. Canada and England, on the other
	hand, created public broadcasting first and only later opened the
	airwaves to commercial stations.\autocite[412]{Meany}

	It was this same era of reform that lead the Carnegie Corporation
	in search of a way to increase the reach of their 
	humanitarian educational programs for inner-city preschoolers.
	\autocite[15]{Davis} Lloyd Morrisett, the vice president of programs at
	the Carnegie Corporation\autocite[7]{Cooney} and a psychologist
	\autocite[15]{Davis} had noticed his own daughter's fascination with
	the television medium when his three-year old got up early in the 
	morning to watch the pre-broadcast RCA test patterns.
	\autocite[11]{Davis}
	
	Morrisett then tasked Joan Cooney, a publicity specialist for the 
	\textit{United States Steel Hour}, a twice-monthly series on CBS, 
	\autocite[27]{Davis} with the project of producing a study
	on the educational possibilities of children's television, backed by a
	\$15,000 grant. The resulting report, \textit{The Potential Uses of 
	Television in Children's Education} analyzed the ways that the
	relatively new technology of television could be used as an educational
	supplement for preschoolers, particularly those from disadvantaged 
	backgrounds.

	It was this report which lead to the creation of the program at hand, 
	\textit{Sesame Street.}

	\section*{Welcome to \textit{Sesame Street}}

	noindent\textit{Sesame Street} needs little introduction. One of the longest
	running shows in the history of television,\autocite{Brit} it has been a 
	cultural staple of generations of children---and their parents---in the 
	United States and worldwide.

	While \textit{Sesame Street} was not the first children's television
	program made to be entertaining to parents, older siblings, or other
	caretakers who may be present---a style which likely originated on 
	\textit{Captain Kangaroo}---it has arguably taken this paradigm the 
	farthest

	While the program's most recognizable faces may be the members of its 
	puppet cast, who hail from Jim Henson's troupe *The Muppets*, music and
	musicians play an important role in the program's educational mission. 

	\textit{Sesame Street} was a welcome addition to the children's media 
	landscape of 1969\footnote{Cite something here}. At that time, there
	was little on television 

	In order to keep up with the non-educational television competition only
	a turn of the dial away, televised preschool education would have to be
	just as entertaining.\autocite[38]{Cooney}

	Basically, \textit{Sesame Street} understands the importance of
	entertainment and attention. 


	\section{Attention}	
	
\newpage
%\singlespacing
\printbibliography
\end{document}
