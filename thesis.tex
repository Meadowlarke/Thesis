\documentclass[12pt,letterpaper]{article}
\usepackage[utf8]{inputenc}
\usepackage[T1]{fontenc}
\usepackage{amsmath}
\usepackage{amsfonts}
\usepackage{amssymb}
\usepackage{graphicx}
\usepackage{setspace}
\usepackage{palatino}
\usepackage[notes,backend=biber]{biblatex-chicago}
\bibliography{thesis.bib}
\usepackage[left=1.20in, right=1.20in, top=1.20in, bottom=1.20in]{geometry}

\begin{document}
	
	\title{Playing \textit{with} the Music: Ecologies of Attention and Understating 
	in the Classical Music Education Project of \textit{Sesame Street}}
	\author{Eden Attar}
	\maketitle
	\doublespacing
	\thispagestyle{empty}
	\newpage
	\clearpage
	\setcounter{page}{1}
	\thispagestyle{empty}

	\section*{Introduction}	

	If I have done anything right this paper should be easy to read. I
	think that is the kindest thing a writer can do. If I have
	done anything right this paper should also not be quite like anything 
	you have seen before. I might say that is the second kindest thing a 
	writer can do.  

	In the spirit of easy reading, here's a summary of what lies ahead:

	This is a paper about the ecologies of attention and understanding 
	evident in and emergent from the presentation of classical music on the
	classic children's television program \textit{Sesame Street.}  
	
	The paper takes three parts. The first part is an analysis of context,
	attention, and education on the program, framed through the analogy of
	bird-watching. I argue that the classical music education which is 
	presented on the program deviates from the humanitarian norm of 
	``uplift'' and absolute aesthetics---the bird's-eye view---and instead
	presents an understanding of music which resonates with ideas of 
	``listening with''---the view of the bird watcher, or in the words of
	Jenny Odell, the bird \textit{listener} who observes and is observed;
	who through acts of attention learns to perceive the world around them
	pluralisticly. In this section I argue that while \textit{Sesame
	Street} engages with aspects of critical pedagogy, it is ultimately 
	resistant to educational classifications. The program is itself
	\textit{ecological}, composed of many interacting and competing parts
	which create a program whose character is emergent and consistent, but
	not unified.   

	Part two is an analysis of classical music performance on the program, 
	framed through the analogy of the worm-listener---they who listen as 
	outsiders to 
	the arcane sound of the world beneath our feet. This section deals with
	questions of parody, virtuosity, and childhood. It contains three
	character studies of classical musicians who made appearances on the
	program: Yo-Yo Ma, Evelyn Glennie, and Lang Lang.   

	In this section I argue that \textit{Sesame Street} takes part in a
	long tradition of classical music parody which welcomes the outsider
	while at the same time engages the classical music native and initiate
	---that is, the worm, who has spent its life embedded in the classical
	strata.
	\textit{Sesame Street} neither challenges nor exalts the classical 
	music canon, but rather serves as an enabling technology, a 
	\textit{geophone}, which allows for the areal listener to perceive and  
	engage with the arcane sounds of underworld.  

	This project takes shape in two media: one is what you are reading 
	right now, the 
	``white paper'' presentation of the research. The other is a 
	non-striated recreation of the field in the form of a Multi User 
	Domain, or MUD. The third section of this paper deals with the themes
	of diversity,
	equity, and liberation which inform the project.  

	Each part begins with a prologue---or dare I say, a dramatic monologue
	---written in first-person perspective
	which grounds the following discussion. If the chapters look at
	musical and educational ecologies, the prologues give lived context for
	the geological bedrock on which these ecologies are based.  

	I'll count myself off: LIGHTS. CAMERA. ACTION!	

	\newpage	
	\section*{Prologue: In \textit{media} res}
	
	\noindent
	On a summer evening in Saint Louis I was listening to the radio in
	the kitchen. In the research for this paper, which had grown from a 
	character
	study on Yo-Yo Ma to encompass the themes of eduction and public 
	broadcasting,
	I had decided that I should actually own some kind of broadcast 
	receiver. 
	After looking into mini televisions, I settled on the conservative 
	option
	of a radio. As I chopped onions, my local NPR station's weekend 
	programming
	buzzed in the background.
	
	A program on the recent events in Afghanistan had ended and the PSA 
	space, where a commercial station
	would have run adds, began. I heard a woman's voice say ``This is Saint
	Louis Public Radio. Understanding Starts Here.'' 

	My ears perked up. What an intriguing declaration. It is not ``This is 
	Saint Louis Public Radio, your source of news,'' or ``This is Saint 
	Louis Public Radio: hear the nation.'' Instead, the 
	slogan is a 
	promise to increase your \textit{understanding. }NPR is offering not
	just to 
	inform you, but moreover to provide some kind of ethical education,
	presumably into cultural-socio-political events such that one comes away
	\textit{understanding.}

	A strange word, ``understanding.'' In the context of the 
	socio-political,
	cultural and economic topics under the purview of national public 
	radio, something seems very kind about the word. It feels emotionally
	laden and steeped in empathy. 

	Its usage does not imply factual understanding, something like ``Let
	us factually explain what's going on in Afghanistan,'' but rather, 
	``cultivate an understanding of the human experiences of Afghanistan 
	along with us.'' 


	Etymologically, the ``under'' in the first half of the word 
	``understand'' does not denote the more modern ``below,'' but rather 
	among or in-between. \autocite{Ety} To understand is to stand among. 
	What \textit{NPR} is offering, then, is empathy in a broadcast. 


	This is continuous with the founding rhetoric of \textit{NPR} and its
	focus on hearing common people speak. 
	\textit{All Things Considered}, now one of the most listened-to programs
	on American radio, debuted with the chaotic coverage of
	what at the time was the biggest anti-Vietnam war protest to date. It
	consisted of tape gathered by multiple reporters at the event with no 
	overarching narrative. It was unlike anything on the radio at the time.
	Its chaos was intended to capture the chaos of the moment, but was also
	part of a larger aesthetic and political trend on the program to hear 
	the voices of normal Americans. Something now known as the 
	``public radio sound.'' 

	The sound is recognizable to this day by its lack of a 
	narrator and the featuring of the voices of common people. It 
	was pioneered by the ``founding mother'' of \textit{NPR}, radio 
	journalist Susan Stamberg. She was inspired by the cinema verité of the
	time, saying that the idea ``was to tell a story without telling it---
	just sound through chunks.'' This meant no authoritative radio voice,
	but instead just giving everyday Americans time to talk before moving 
	along to another speaker. The work of narrative interpretation is left
	to the listener, who is given a more rhizomatic view of the events---
	a radical new direction in radio journalism and broadcast culture more
	generally.  

	One particular aspect of the public radio sound was the use of ``cross
	talk,'' a technique in which two voices are layered onto one another 	
	in polyphony, which was inspired by Glenn Gould's ``contrapuntal 
	storytelling'' in \textit{The Solitude Trilogy.}\autocite[197]{Porter}
	Stamberg retrospectively described the debut as having ``no interfering 
	narrator.''\autocite[185]{Porter}

	The showcasing of alternative voices was so central to the ethos of the
	young network that \textit{NPR} turned down a \$300,000 donation from 
	the Ford 
	Foundation contingent on the network hiring well-known broadcast
	journalist Edward P. Morgan as host of \textit{All Things Considered}. 
	Morgan, recipient of a Peabody Award, would have certainly 
	brought the new network
	credibility, but \textit{NPR} feared that the authoritative voice would
	get in the way of the alternative voices that they wanted to showcase.
	The ethos of ``standing among'' seems to be baked in to the entire
	project of public radio.

	This ethos is visible in public television as well. In the mid-1960s
	Ralph Lowell, Board Chairman of the Boston Educational Television 
	station proposed that the Oval Office put together a commission on 
	educational television. President Johnson, believing that this was
	a job for the private sector, turned to the Carnegie Corporation.
	What was formed was a diverse team known as the Carnegie Commission
	on Educational Television. Among others, this team included writer Ralph
	Ellison, concert pianist Rudolph Serkin, and labor leader Leonard 
	Woodcock.\autocite[409]{Meany}

	The commission put together a report titled \textit{Public Television, A 
	Program for Action} which stressed the artistic and transformative value
	of the medium and the importance of using it to resist the pressure 
	toward uniformity found in commercial television but instead seek out 
	and satisfy the needs of the nation's diversity.\autocite[410]{Meany}

	The need for this shift in broadcasting came about because the
	United States Government had granted access to the airwaves to 
	commercial stations
	first, only adding public stations to the media landscape as part of
	Great Society era reformations. Canada and England, on the other
	hand, created public broadcasting first and only later opened the
	airwaves to commercial stations.\autocite[412]{Meany}

	It was this same era of reform that lead the Carnegie Corporation
	in search of a way to increase the reach of their 
	humanitarian educational programs for inner-city preschoolers.
	\autocite[15]{Davis} Lloyd Morrisett, the vice president of programs at
	the Carnegie Corporation\autocite[7]{Cooney} and a psychologist
	\autocite[15]{Davis} had noticed his own daughter's fascination with
	the television medium when his three-year old got up early in the 
	morning to watch the pre-broadcast RCA test patterns.
	\autocite[11]{Davis}
	
	Morrisett then tasked Joan Ganz Cooney, a publicity specialist for the 
	\textit{United States Steel Hour}, a twice-monthly series on CBS, 
	\autocite[27]{Davis} with the project of producing a study
	on the educational possibilities of children's television, backed by a
	\$15,000 grant. The resulting report, \textit{The Potential Uses of 
	Television in Children's Education} analyzed the ways that the
	relatively new technology of television could be used as an educational
	supplement for preschoolers, particularly those from disadvantaged 
	backgrounds.

	It was this report which lead to the creation of the program at hand, 
	\textit{Sesame Street.}


	\newpage
	\thispagestyle{empty}
	\vspace*{30pt}
	\begin{center}
	{\Huge Part I\\ 
	\Large The Bird-Watcher:\\
	Attention, Education, and Entertainment}
	\end{center}
	\newpage

	\section*{Entertainment: Welcome to \textit{Sesame Street}}

	\noindent\textit{Sesame Street} needs little introduction. One of the
	longest
	running shows in the history of television,\autocite{Brit} it has been a 
	cultural staple of generations of children---and their parents---in the 
	United States and worldwide.
	
	While \textit{Sesame Street} is an educational program, it is notable 
	for its entertainment value. This stems from the realization by the 
	original producers of the program that in order to compete with the 
	non-educational television competition only a turn of the dial away,
	televised preschool education would have to be just as 
	entertaining.\autocite[38]{Cooney} This is especially true given that
	at the program's premiere in 1969, there was nothing like it in the 
	children's media landscape.\autocite[See chapter 3 for a survey of the 
	contemporary children's television landscape]{Davis} This means that 
	there were no models for the production of the program, nor for its
	reception---they could not trust that children knew how to watch
	educational programs.  

	This realization lead to a fundamental characteristic of the production
	structure of the program: rather than hire child development and 
	education specialists to produce the show, they used education research
	to inform decisions made by seasoned TV, radio, and Broadway producers,
	writers, composers, and actors. This decision was informed by the 
	observation
	that children respond well to adult television and were known to recite
	adds for beer and cigarettes, as well as having their attention held by
	adult variety shows such as \textit{Rowan and Martin's Laugh-In.}
	\autocite[16]{Ostrofsky2012}

	But \textit{Sesame Street} is not just entertaining for children. 
	In order to encourage parents to watch the show along with their
	children, a move which would also help keep preschoolers focused,
	the producers aimed to make the show entertaining enough for parents
	to want to watch it along with their children.
	\autocite[294]{Ostrofsky2017}
	\footnote{Mixed-age television targeting likely began on the 
	television program \textit{Captain Kangaroo}. \autocite[46]{Davis}} 
	
	In terms of music, you see to this effect the featuring of musical 
	performers that parents would know, often playing educational parodies 
	of their own work. For example, Norah Jones...
		
	Entertainment, the producers understood, is a tool for
	cultivating attention.
	

	\section*{Listening with: Attention as Grounds for Ethics}	

	The philosophy of music on \textit{Sesame Street} was perhaps best 
	summarized by Joe Rapso, the first music director of the program, in a
	1971 press release. The program's diverse music, he said, would
	help bridge cultural divides when the kids were grown up and ``bring all
	kids together, whether they live in Grosse Point or on 148th Street in 
	the Bronx... And the beauty of our music is maybe that the child in the
	Grosse Pointe home is hearing gospel and blues for the first time and 
	the black child in the urban ghetto is hearing a harpsichord and flute 
	for the first time. Someday, when they grow up, they'll have one more 
	thing in common.''\autocite[297]{Ostrofsky2012}

	By using entertainment to focus attention and expose children to music 
	from other social and economic contexts,
	Rapso is promising not only a musical education, but also an ethical 
	one. Having ``one more thing in common'' is perhaps here grounds for 
	the social understanding which is historically valued by public media.  

	This understanding through sonic exposure resonates with the theories 
	of media developed by Jenny Odell in 
	\textit{How To Do Nothing}, a book at the intersection of media theory
	and ecology. Odell places attention as the center of ethics. We decide 
	who is seen, who is heard, and who has agency, through acts of 
	attention. \autocite[154]{Odell}

	Echoing Pauline Oliveros, Odell describes a way of engaging with the 
	world similar to that of bird watching, which she says is better 
	described as bird \textit{listening}.
	By listening with the birds---listening in a way which acknowledges
	ones own subjectivity and their location as another sounding element of
	the sonic world---the listener learns to differentiate and identify 
	the sounds 
	of the different birds in an environment, and thereby becomes aware
	of the ecology around them and their place within it.\autocite{Odell} 
	
	This ecological, attention-based understanding of the world requires 
	that one gives up the idea of discrete entities, simple origin stories,
	and simple one-to-one causalities, and above all, requires time.
	The direct product of this attention, Odell says, is context. 
	That is, paying close attention can help us better understand the 
	nuanced ecologies of being and identities that make up the ecological
	relationships of an environment. The longer that the attention is held, 
	the more context appears.\autocite[155]{Odell} 

	This is musically apparent in the skits of Muppet Simon Soundman.
	In one skit, Simon visits a music store, where he tell the shopkeeper
	that he is in the market for a ``nice, shiny---'' and then we hear a
	long trumpet lick as his mouth moves in time. 

	``Uhh... Ohh... Would you mind repeating that, sir?''

	``No, indeed. What I said was I'd like to buy a nice, shiny---'' and
	then the trumpet sounds again. 

	The shopkeeper goes off-screen and returns with an instrument. 
	 
	``Here you are, sir. A beautiful new violin.''

	``No, no, no... I don't want a---'' this time he makes violin
	sounds.  ``You see, because what I asked for was a nice, shiny---''
	
	And the skit goes on like this with for a number of minutes with the 
	shopkeeper bringing back the wrong instruments and Simon Soundman
	referring to them only through impeccable renditions of their sound. 
	
	This skit uses a simple and entertaining gag to expose children to the
	sounds, names, and appearances of several common instruments. Like Odell
	discusses leaning to identify birds by their sounds, children are	
	taught here to learn to identify the sounds of several common 
	instruments that you would find in the environment of music store or 
	a big band---the sound that the shopkeeper himself makes once Simon
	leaves, saying ``that fellow was pretty good. Huh. I should have asked
	him if he would like to play in our---.''  

	But Rapso's statement on music goes beyond awareness and context of 
	musical instruments. It is primarily one of social context. To quote 
	a pioneering broadcast journalist Edward R Murrow,
	the role of television is to make the world aware of
	itself.\autocite[54]{Davis} By creating a media platform through which 
	children of different backgrounds can gain exposure to each other's 
	music, they
	will have an increased awareness and understanding of each other.
	Implicit in all this is the idea that this will relive social and class
	tensions. Rapso was implying a social and ethical education based in
	context and understanding through attention. 


	This attitude, the social context of listening, is apparent in a number 
	of musical skits on the program.  


	\begin{quote}
	\subsubsection*{Case Study 1: Itzahk Pearlman and ``What's Easy for You
	 is Hard for Me''}

	Fading in from black, there is a simple stage set with three folding 
	chairs. On the third to the right is an instrument case. Entering from 
	the bottom of the frame, a girl runs up the big grey steps to the stage 
	and takes a seat to the left, smiling. She looks in the direction she 
	came from. Itzhak Pearlman enters with a cane in each hand. He begins 
	to climb the steps to the stage, swinging each stiff leg up to the next
	level and climbing one step at a time. 

	There are three big steps, presumably made from grey plywood, and then a
	small step up to the stage. By the second step, the girl begins to 
	fidget. By the third step her smile has faded and she looks to the side
	and kick her feet while she waits, perhaps to avoid staring.   

	Pearlman climbs the small remaining distance between the steps and the 
	stage and turns to sit down. 

	``Ooff. Those steps,'' he says.

	In order to bend at the waist and hips to sit, he unlocks his leg 
	braces with an audible click and a rotation of his hips. He sits down 
	and pulls his hands out of the metal rings of his canes.

	``You know,'' he says, turning to the girl, ``some things that are 
	really easy for you are hard for me.''

	He picks up his violin and checks the tuning with two quick plucks with 
	his fingerboard hand and then a quick bow across each adjacent string
	pair, eyes closed. His tone is rich and bright, even when tuning. He 
	launches into an impressive flourish, starting with double stops 
	followed by a long run up and down the fingerboard, ending with a right
	hand pizzicato chord. He does this all with a certain light 
	effortlessness.

	``Yeah, but some things are easy for you that are hard for me,'' the 
	girl replies. She lifts her violin and plays a minor melody. She 
	carries the tune, but her intonation is unstable. 

	Pearlman tilts his head and leans in with appreciation as she plays.
	They both smile when she finishes.   

	\end{quote}

	The view of classical music which we gain access to here is not a
	traditional one. This is not a performance for an audience, but instead
	a intimate moment of discourse and vulnerability between two 
	musicians. Here, classical music is not seen from the birds eye view,
	that of 
	objective distance, looking over the musical landscape, but that of the 
	bird listener listening along with. There is no audience in this skit, 
	it is
	more about the interaction of the two musicians and how they listen to 
	and acknowledge each other's lived experiences of music and the world. 
	We may be watching from home as a television audience, bit what we see
	is not the somber performance of canonic works, nor is it an accessible
	presentation of a classical music program for non-initiates. What we 
	are watching is a demonstration of ecological musical listening, of 
	listening with. 

	One might object that this is \textit{media}, and that what we are
	seeing is fake, a construction. That is, in fact, the case. However, 
	that does not reduce the potential of an ecological reading of this 
	skit, nor its efficacy as a demonstration. What we are seeing here is a 
	representation of an empathetic and ecological experience and a 
	sandboxed recreation of real life, inasmuch as the producers created an
	actual experience on the set which we can view the recording of---an 
	experience which speaks to actual, lived, human encounters. 
	
	This encounter is no less real for being filmed. Pearlman actually 
	limped up the tall steps onto the stage as he had done countless times 
	outside of the \textit{Sesame Street} set---this time was no easier than
	any other. The girl actually played as she has obviously done for
	hundreds of hours, and this time was no easier than any other. 
	Though what we at home are seeing is a representation of a social 
	encounter, it is a recreation of actual	experiences. 

	And its sandboxing does not detract from its meaningfulness, but rather,
	it is only through this sandboxing that it can be made into 
	entertainment and
	education such that it can contribute to an ethical education. 
	
	I say all of this to argue that an ecological reading and ecological 
	education 
	can come out of ``constructed'' media, that ultimately, it is not 
	un-reality which is created, but a guided slice of reality, made in 
	order to facilitate the ecological goal of awareness and context.  
	
	This next skit similarly features a disabled artist, deaf percussionist 
	Evelyn Glennie. However, while the Perlman skit called attention to
	ability and disability, the following skit chooses not to address 
	disability directly and instead focuses on 
	on musical and social listening in the context of a classical music
	performance. 
	
	\begin{quote}
	
	\subsubsection*{Case Study 2: Glennie and Linda}
	
	Glennie is first introduced by an off-screen voice: ``And Now, Evelyn 
	Glennie, world famous percussionist, will perform a duet with her 
	friend Linda.''

	Glennie walks in followed by Linda.  

	``Now, I'll tell \textit{you} when to come in. Okay," Glennie says, 
	signing and 
	speaking simultaneously, and hands Linda a single mallet.''  

	Glennie launches into an impressive piece of solo classical marimba for 
	four mallets. 

	At the beginning of an impressive run, Linda taps Glennie's shoulder.

	``Now?'' she signs and mouths.

	``Not now, not now,'' Glennie responds, lifting her mallets. She continues
	 with the run down from the high end of the Marimba to the middle. A 
	small hiccup at the end of the phase suggest that she is slightly 
	flustered, but only slightly. She continues on to the next phrase, 
	displacing Linda, who stands by her side watching closely, which each 
	step as she moves up and down the instrument. With a massive run upwards
	she finishes the phrase and raises her mallets into the air. Linda taps 
	her shoulder.

	``\textit{Now?}'' 

	``Not now,'' Glennie shakes her head. 

	Glennie crosses the stage to an assortment of unhitched percussion 
	instruments, including a snare drum and numerous cymbals and other 
	pieces of resonant metal. Glennie plays a very modern sounding solo, 
	impressive in its wash of timbres and coordination. At first Linda 
	plays close attention, but after a few moments she starts to look bored 
	and disappointed, crosses her arms, and lets her head droop.

	Glennie plays a roll on the snare and looks at her partner expectantly. 
	Linda is not listening. She tries again. The response is the same. 
	Glennie taps the splash cymbal lightly. She pokes Linda's shoulder. She 
	looks up.

	``Now!'' Glennie gestures at the cymbal with a large gesture. 

	Surprised, but excited, Linda raises her mallet and strikes the cymbal. 
	The audience breaks out in applause. The two look at each other with a 
	smile and shake hands. 

	They walk to the center stage and take a bow, first to the camera---
	representing the concert-hall audience---and then to each other. A girl 
	walks up to them and hands Linda, the amateur, a bouquet of flowers. 
	Glennie watches, miffed. She puts her hands on her hips and with a look 
	of shock and disappointment on her face. Linda, soaking up the applause, 
	smiles, and hands Glennie back her mallet, a gesture much like the lead 
	ballerina pulling a single rose out of her bouquet to hand to her 
	partner. Glennie takes the mallet, and storms off the stage, frowning. 
	Linda turns back to camera and smiles.   
	
	\end{quote}

	Whereas the first skit centered on ability and disability, this skit 
	does not present the two deaf performers as encountering any difficulty,
	not including interpersonal relationships. This is, in fact, where the
	humor of this skit comes from: their inability to listen to each other
	and their resultant misunderstanding of the meaning of a duet. There is
	a comic symmetry to their misunderstanding: Glennie is unfair to Linda, 
	during the performance, and Linda returns the favor at the end. 

	By focusing on this misunderstanding, deaf interaction is normalized,
	attention is drawn away from disability to the point that one might not
	even realize that the two performers are deaf---not in an erasure of 
	disability, but through the showcasing of Glennie's musical virtuosity,
	her domination of the stage, and the skit's focus on the friction
	which this creates between her and Linda.  

	This friction presents, as part of the musical lesson of the skit, 
	what a duet is \textit{not}, and in so doing suggests more fair ways
	that children might play with each other, musically or non-musically.
	This aspect of play is an enduring aspect of the musical performances
	on \textit{Sesame Street}, where adults often act in childlike ways.
	Analogs for children, Glennie and Linda show what can go wrong if we
	fail to listen---however it is in all of our human diversity that we
	do so. This skit presents an ethics of attention based not only in
	listening, but also in giving space for others to speak, make sound,
	and communicate. Through a comical demonstration of misunderstanding, 
	it delivers a lesson on how understanding starts from fair and balanced
	communication.	


	\section*{Education}	
	

	\textit{Sesame Street} was met with a completely heterogeneous 
	appraisal of its educational model after its debut. 

	Echoing the thoughts of many traditional educators, preschool authority
	Carl Bereiter of the Ontario Institute for Studies in education wrote in
	the \textit{Wall Street Journal} the morning after the program's 
	premiere that the program was too far removed from ``structured'' 
	teaching and warned that the program could fail because ``it's based 
	entirely on audience appeal and is not really teaching anything in 
	particular.''\autocite[200]{Davis}

	It received an equal amount of criticism form progressive educators, 
	some of who thought that the program was too traditional. Frank 
	Garfunkle, a professor of education at Boston University wrote in the 
	alumni magazine that ``any claim that \textit{Sesame Street} is a major 
	educational or media innovation is preposterous. The values implicit in
	the form and content--strictly three Rs with a mixed bag of dressing---	
	are traditional... The image of \textit{Sesame Street} as a unique 
	vanguard of educational experience is a mirage.''\autocite[201]{Davis}

	When the program was later slotted to be shipped abroad in 
	international co-productions, the BBC turned it down. Children's 
	program chief, Monica Sims, described the program ``authoritarian'' 
	referring to its aims to change children's behavior and its emphasis on 
	right answers.\autocite[211]{Davis} This last observation is at odds
	with the intentions stated by Joan Ganz Cooney in her inaugural
	research for the program: almost all of the experts she interviewed 
	wanted to see the teaching of cognitive habits---as defined by Jerome 
	Kagan to be analysis, generating hypotheses, and reflection---rather 
	than trying to teach them skills such as how to read. The idea was to
	teach how to think, not what to think.\autocite[23]{Cooney}

	That there were such varied responses to the educational model of the
	program speaks to its complexity and its resistance to educational 
	labels. 
	
	On the face of it, \textit{Sesame Street} quite literally conforms to 
	the broadcast model of education. The whole project is, after all, 
	inherently instructivist: it was created as a response to the question
	``how can we use new technology to give preschool education to as many 
	children as possible?'' The program 
	uni-directionally streams educational material to the learner watching 
	and listening at home, who is left to absorb the information. The 
	learner has no say in constructing their own education. When it comes to
	the educational material it presents on basic literacy and numeracy 
	there are, as noted by the BBC, clear right and wring answers. 
	Furthermore, the program has a curriculum 
	designed by education researchers with intended outcomes and takeaways 
	for each lesson.\autocite[117]{Davis} 

	Literacy and numeracy education are, however, only part of the 
	educational project of \textit{Sesame Street}. A great deal of the 
	programming can be interpreted as, and has been expressed by its 
	creators, as giving kids a cultural and emotional 
	education.\autocite[297]{Ostrofsky2012}
	A great deal of this cultural and emotional education is open ended. 
	In the skit with Itzahk Pearlman, viewers are exposed to the stunning 
	playing of a professional, the attainable playing of a child, 
	disability, ability, and co-mutual visibility. When the two musicians 
	play for one another, they do so to express their mutual relationship, 
	not simply so that the other may listen. They are both listening and 
	playing with. 

	What is a child at home supposed to get from this? This is where 
	\textit{Sesame Street} may be instructivist and broadcast-based, but 
	is moreover \textit{non-striated.} The listening with aspect of its 
	educational model drops the learner into an existing network of 
	understanding, discourse, and culture, such that what they take away
	from the lesson depends on their personal proclivities and needs, 
	whether that is simply hearing the master violinist play, or the 
	understanding that some thing that is easy for one person can be hard 
	for others.

	I say all of this not to make the hard-line argument that \textit{Sesame
	Street} is virtuous and non-traditional. Rather, I am attempting to 
	demonstrate how resistant to classifications the \textit{Sesame Street}
	curriculum is. It is somehow at once instructivist yet open-ended. I 
	argue that this is in no small part due to its non-striated character, 
	where learners are free to interpret the skits to find whatever 
	emotional or cultural meaning that they themselves need.









	\newpage
	\thispagestyle{empty}
	\vspace*{30pt}
	\begin{center}
	{\Huge Part II\\ 
	\Large The Worm-Listener:\\
	Parody, Virtuosity, and Childhood}
	\end{center}


	\newpage
	\section*{Prologue: The Worm Listener}

	I spent something like a quarter of my life on the south side of 
	Chicago. There was a summer during that period where I got into the 
	habit of biking down to the 63rd street beach to comb the sandy shore 
	for pebbles. There were two kinds that interested me: shiny black 
	pieces of anthracite coal, and pale grey honeycombs of steel slag. Some
	thirty miles down the coast, at what now is the Indiana Dunes National 
	Park, the beach is littered with pebbles of all kinds, glittering 
	happily beneath the surf or ice of the ancient lake. Here, however, in 
	the scorched earth left behind by the bonfire that built and destroyed 
	America, the pebbles were largely fossils of that recent past.

	That summer I went on a sound walk organized by the Midwest Society of
	Acoustic Ecology at the former site of the US Steel foundry. A section 
	of the massive industrial campus had been re-wilded and returned to its
	pre-steel state of fresh-water marsh. Some birds chirped, we saw an 
	egret, but it had turned out to be one of those hot summer days where 
	the air is still, heavy, and all the more stifling for its enveloping 
	silence. 

	At some point the walk took us to a part of the wooded hammock where 
	the artists had driven microphones---geophones---into the earth. When 
	my turn came around I put on the headphones and listened. 

	I was immediately made aware of my on weight on the ground beneath me. 
	Any small motion and I would hear myself in the microphone like an 
	earthquake. Any large motion and the bugs, which I now realized were all
	around me, would stop and I would have to wait for some time before they
	started to move again. 

	I had never heard anything like it. It was a sound impossible to 
	describe in terms of our air-based hearing. And I was part of it. The 
	subterranean bugs heard and responded to me just as I to them. Through 
	the assistive technology of contact microphones and digital amplifiers, 
	I was given access to the worm's-ear view of the ever-present world 
	beneath my feet. The sounds were foreign, but in no way unpleasant. Is 
	this the sound of an ant, a worm? I don't know, but I like it. For me as
	an air-based listener, this was the sonic experience of a complete outsider.	
	
	\bigskip

	\begin{center}
	***
	\end{center}

	\noindent The Bernstein Young Peoples Concerts...
	
	\newpage
	\section*{Listening \textit{with} the Music: Ecological Humor} 

	
\newpage
%\singlespacing
\printbibliography
\end{document}


%	While the program's most recognizable faces may be the members of its 
%	puppet cast, who hail from Jim Henson's troupe *The Muppets*, music and
%	musicians play an important role in the program's educational mission,
%	as well as its multi-age appeal. 


